% -*- latex -*-

\documentclass{article}

%% packages %%

\usepackage{amssymb}
\usepackage{amsmath}
\usepackage{hyperref}
\usepackage{unicode-math}

%% /packages %%

%% Marginal notes %%

\usepackage{xargs}
\usepackage[colorinlistoftodos,prependcaption,textsize=tiny]{todonotes}
\newcommandx{\unsure}[2][1=]{\todo[linecolor=red,backgroundcolor=red!25,bordercolor=red,#1]{#2}}
\newcommandx{\info}[2][1=]{\todo[linecolor=green,backgroundcolor=green!25,bordercolor=green,#1]{#2}}
\newcommandx{\change}[2][1=]{\todo[linecolor=blue,backgroundcolor=blue!25,bordercolor=blue,#1]{#2}}
\newcommandx{\inconsistent}[2][1=]{\todo[linecolor=blue,backgroundcolor=blue!25,bordercolor=red,#1]{#2}}
\newcommandx{\critical}[2][1=]{\todo[linecolor=blue,backgroundcolor=blue!25,bordercolor=red,#1]{#2}}
\newcommand{\improvement}[1]{\todo[linecolor=pink,backgroundcolor=pink!25,bordercolor=pink]{#1}}
\newcommandx{\resolved}[2][1=]{\todo[linecolor=OliveGreen,backgroundcolor=OliveGreen!25,bordercolor=OliveGreen,#1]{#2}} % use this to mark a resolved question

%% /Marginal notes %%

%% macros %%

\newenvironment{record}{\begin{description}}{\end{description}}

\newcommand{\idsof}[1]{\mathcal{I}_#1}
\newcommand{\id}{\mathsf{id}}
\newcommand{\blockids}{\idsof{\mathcal{B}}}
\newcommand{\agentids}{\idsof{\mathcal{A}}}
\newcommand{\slotleader}{\mathsf{sl}}

%% /macros %%

\title{Rule-based specification of the blockchain logic}
%\author{}

\begin{document}

\maketitle

\improvement{At the moment, we ignore multi-party communication, but
  we may have to add some rules related to them in the document.}

This document describes the rules underpinning the blockchain data
structure of Cardano's settlement layer. These rules are described
independently of any network or communication mechanism: it is assumed
that some mechanism provides us with, \emph{e.g.}, blocks, and that
our task is to assume that they are accepted or not. Notably, we make
no difference between rejecting a block and storing it for later use.

A direct consequence of ignoring the communication mechanisms, is that
we do not have to serialise data, hence data is represented in
abstract syntax, with no reference to a binary representation.

As a further simplification, we ignore any unknown data, unparsed
parameters, and versioning issue. This document specifies the system
as it is.\improvement{We may want to add something about protocol
  updates, though.}

\section{Definitions}

\info{General definitions used in all sections. Like the
  slot-to-leader function, the global-time-to-slot function, the
  abstract syntax of transactions and of blocks, \ldots}

\begin{description}
\item[System of identifiers] A set $A$ is said to \emph{have
    identifiers} if there is a set, written $\idsof{A}$ and an
  injective function $\id ∈ A ⟶ \idsof{A}$. In implementations,
  identifiers are typically realised by cryptographic hash functions,
  which are injective for all intent and purposes.

  When an item $a∈A$ such that $\id(a) = h$ is clear from the context,
  we will often identify $a$ and $h$. Similarly, when given an $a∈A$
  and a $h∈\idsof{A}$,

\item[Agent id] We assume a set $\agentids$ of identifiers
  representing the agents interacting on the blockchain. In an
  implementation agent ids would be public cryptographic keys.

\item[Slot] An integer.\improvement{I'm ignoring epochs for the time
    being}

\item[Slot leader] We assume known a function
  $\slotleader ∈ \mathbb{N} ⟶ \agentids$ mapping each slot to an agent
  id.\improvement{This function is actually also parametrised by a valid
    blockchain. And doesn't look arbitrarily far into the future.}

\item[Current slot] The current slot is a parameter of every
  verification (\emph{i.e.} it doesn't make sense in general to ask
  whether a chain is valid, only that it is valid \emph{at the current
    slot}). In an implementation, the current slot would be computed
  from the clock time. The role of the current slot is to prevent
  attacks where a malicious slot leader at slot $i+1$ would try to
  issue a block without acknowledging the block issued by the slot
  leader at slot $i$.

\item[Transaction] \improvement{Todo}

\item[Block] Blocks have a system of identifier $\blockids$. There is
  a distinguished $b_0∈ \blockids$, called the genesis identifier,
  which is not in the image of $\id$.

  A block
  consists of\improvement{I'm ignoring epoch boundary blocks, and the
    genesis block. Genesis block is not really important, but epoch
    boundary blocks may become relevant for this specification}
  \begin{description}
  \item[predecessor] A block identifier
  \item[slot number] The slot at which the block has been issued
  \item[issuer] The identifier of the agent who issued this block. In
    an implementation, this would be realised by a cryptographic
    signature.
  \item[transactions] A list of transactions\improvement{Ignored:
      secret-sharing, updates, delegation. Eventually we need at least
    delegation}
  \end{description}

\item[Chain] A chain is a sequence of blocks $b₁,…,b_n$ such that the
  predecessor of $b₁$ is the genesis block, and for every $i$, the
  predecessor of $b_{i+1}$ is $b_i$.
\end{description}

\section{Adding a block}

\info{In this section we are interested in specifying what it means to
  add a block, given by other means, on top of some existing
  chain. The state is a single chain (aka list). The block must be
  added on top (aka in front) of it.}

A chain\improvement{We can replace ``chain'' by sequence and put the
  predecessor condition in the validity rule. Though this opens a
  question: from the valid extension rules, can I directly define the
  blockchain's forest data structure?} $b₁,…,b_n$ is said to be
\emph{valid} if $b₁,…,b_{n-1}$ is valid, and $b_n$ is a \emph{valid
  extension} of $b₁,…,b_{n-1}$.

A block $b$ is said to be a valid extension of a valid chain
$b_1,…,b_n$ if all of the following hold
\begin{description}
\item[Predecessor] The predecessor of $b$ is $b_n$.
\item[Not future slot] The slot number is no larger than the current
  slot.
\item[Issuer] The issuer is the slot leader at the block's slot
  number\improvement{This ignores delegation}.
\item[TODO]
\end{description}

\section{Extending the blockchain}

\info{In this section we will specify the rules used by a slot-leader
  to create a new block. The state is a single chain, plus outstanding
  transactions.}

\section{Entering the mempool}

\info{In this section we want to specify how individual transactions
  are added to or rejected from the mempool. Might have to be moved
  before the blockchain-extension rules. The state is a single chain,
  plus outstanding transactions.}

\section{Selecting the longest chain}

\info{This section is concerned with the purpose of (locally) deciding
  which chain is the longest (\emph{e.g.} if I'm a slot leader, which
  chain I'm going to extend) among all possible chains produced by the
  network. The state is a tree of chains rooted in the origin block.}

\end{document}
