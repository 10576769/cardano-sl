% -*- latex -*-

\documentclass{article}

%% packages %%

\usepackage{amssymb}
\usepackage{amsmath}
\usepackage{hyperref}
\usepackage{unicode-math}

%% /packages %%

%% Marginal notes %%

\usepackage{xargs}
\usepackage[colorinlistoftodos,prependcaption,textsize=tiny]{todonotes}
\newcommandx{\unsure}[2][1=]{\todo[linecolor=red,backgroundcolor=red!25,bordercolor=red,#1]{#2}}
\newcommandx{\info}[2][1=]{\todo[linecolor=green,backgroundcolor=green!25,bordercolor=green,#1]{#2}}
\newcommandx{\change}[2][1=]{\todo[linecolor=blue,backgroundcolor=blue!25,bordercolor=blue,#1]{#2}}
\newcommandx{\inconsistent}[2][1=]{\todo[linecolor=blue,backgroundcolor=blue!25,bordercolor=red,#1]{#2}}
\newcommandx{\critical}[2][1=]{\todo[linecolor=blue,backgroundcolor=blue!25,bordercolor=red,#1]{#2}}
\newcommand{\improvement}[1]{\todo[linecolor=pink,backgroundcolor=pink!25,bordercolor=pink]{#1}}
\newcommandx{\resolved}[2][1=]{\todo[linecolor=OliveGreen,backgroundcolor=OliveGreen!25,bordercolor=OliveGreen,#1]{#2}} % use this to mark a resolved question

%% /Marginal notes %%

%% macros %%

%% /macros %%

\title{Rule-based specification of the blockchain logic}
%\author{}

\begin{document}

\maketitle

\improvement{At the moment, we ignore multi-party communication, but
  we may have to add some rules related to them in the document.}

This document describes the rules underpinning the blockchain data
structure of Cardano's settlement layer. These rules are described
independently of any network or communication mechanism: it is assumed
that some mechanism provides us with, \emph{e.g.}, blocks, and that
our task is to assume that they are accepted or not. Notably, we make
no difference between rejecting a block and storing it for later use.

\section{Definitions}

\info{General definitions used in all sections. Like the
  slot-to-leader function, the global-time-to-slot function, the
  abstract syntax of transactions and of blocks, \ldots}

\section{Adding a block}

\info{In this section we are interested in specifying what it means to
  add a block, given by other means, on top of some existing
  chain. The state is a single chain (aka list). The block must be
  added on top (aka in front) of it.}

\section{Extending the blockchain}

\info{In this section we will specify the rules used by a slot-leader
  to create a new block. The state is a single chain, plus outstanding
  transactions. }

\section{Entering the mempool}

\info{In this section we want to specify how individual transactions
  are added to or rejected from the mempool. Might have to be moved
  before the blockchain-extension rules. The state is a single chain,
  plus outstanding transactions.}

\section{Selecting the longest chain}

\info{This section is concerned with the purpose of (locally) deciding
  which chain is the longest (\emph{e.g.} if I'm a slot leader, which
  chain I'm going to extend) among all possible chains produced by the
  network. The state is a tree of chains rooted in the origin block.}

\end{document}
